\documentclass[jair,twoside,11pt,theapa]{article}
\usepackage{jair, theapa, rawfonts}

\jairheading{1}{2018}{1-15}{8/17}{12/17}
\ShortHeadings{Reinforcement Learning of Natural Language}
{Uc-Cetina, Navarro-Guerrero, Martin-Gonzalez, Hinaut, \& Wermter}
\firstpageno{25}

\begin{document}

\title{Reinforcement Learning for Natural Language Processing}

\author{\name V\'ictor Uc-Cetina \email uccetina@correo.uady.mx \\
       \addr Computational Learning and Imaging Research Laboratory, Facultad de Matem\'aticas, \\
       Universidad Aut\'onoma de Yucat\'an, Anillo Perif\'erico Norte, C.P. 97119, M\'erida, Mexico
       \AND
       \name Nicol\'as Navarro-Guerrero \email navarro@informatik.uni-hamburg.de \\
       \addr Knowledge Technology Group, University of Hamburg, Department of Computer Science,\\ 
       Vogt-Koelln-Str. 30, 22527 Hamburg, Germany
       \AND
       \name Anabel Martin-Gonzalez \email amarting@correo.uady.mx \\
       \addr Computational Learning and Imaging Research Laboratory, Facultad de Matem\'aticas, \\
       Universidad Aut\'onoma de Yucat\'an, Anillo Perif\'erico Norte, C.P. 97119, M\'erida, Mexico
       \AND
       \name Xavier Hinaut \email xavier.hinaut@inria.fr \\
       \addr Inria Bordeaux, Sud-Ouest Research Centre, 200, \\
       avenue de la Vieille Tour 33400, France
       \AND
       \name Stefan Wermter \email wermter@informatik.uni-hamburg.de \\
       \addr Knowledge Technology Group, University of Hamburg, Department of Computer Science,\\ 
       Vogt-Koelln-Str. 30, 22527 Hamburg, Germany}

% For research notes, remove the comment character in the line below.
% \researchnote

\maketitle


\begin{abstract}
Reinforcement learning algorithms do not seem to be the kind of tool that we would associate to natural language 
processing tasks, at least, not in a first thought. However, in the last decades many researchers have
explored the use of reinforcement learning as one main component in the solution of specific natural language tasks. 
This paper review the state of the art of reinforcement learning methods applied
to solve different problems of current interest in natural language processing.
We analyzed why for certain problems reinforcement learning algorithms have provided
successful solutions and for which others they are not appropriate. Finally, we point out which other problems  
from natural language processing might benefit from reinforcement learning in the near future.
\end{abstract}

\section{Introduction}
\label{Introduction}

\section{Analysis of Reinforcement Learning and Natural Language Processing}
\label{Analysis}
%analyze a significant body of AI research and make it more accessible to a broader audience.
We analyze different natural language processing tasks and the reinforcement learning algorithms used to solve them.

\section{Context and Impact }
\label{Context}
%position existing research results in a broader context and explain their impact; 

\section{Research Gaps Needed to be Addressed}
\label{Gaps}
%identify deficiencies or gaps in current knowledge that need to be addressed in future research. 

\section{New Research Directions}
\label{Research}
%bring together previously unconnected lines of research in a way that fosters new research directions in these areas; 

\section{Conclusions}
\label{Conclusions}

\cite{Levin1997}
\cite{Singh1999}
\cite{Walker2000}
\cite{Oh2000}
\cite{Singh2000}
\cite{Levin2000}
\cite{Ratnaparkhi2002}
\cite{Singh2002}
\cite{Schatzmann2006}
\cite{Young2010}
\cite{Vogel2010}
\cite{Dethlefs2011a}
\cite{Dethlefs2011b}
\cite{Branavan2012}
\cite{Banchs2012}
\cite{Gasic2013a}
\cite{Gasic2013b}
\cite{Young2013}
\cite{Nio2014}
\cite{Gasic2014}
\cite{Guo2015}
\cite{Narasimhan2015}
\cite{Li2016}
\cite{He2016}
  

\vskip 0.2in
\bibliography{sample}
\bibliographystyle{theapa}

\end{document}






